% !TEX encoding = UTF-8 Unicode

\documentclass[a4paper]{article}

\usepackage{color}
\usepackage{url}
\usepackage[utf8]{inputenc} 


\usepackage[english,serbian]{babel}

\usepackage[unicode]{hyperref}
\hypersetup{colorlinks,citecolor=green,filecolor=green,linkcolor=blue,urlcolor=blue}


%\addbibresource{Literatura.bib} 
\begin{document}

\title{Da li Veštačka inteligencija može da zameni ljudsko postojanje ?\\ 
\small{Seminarski rad u okviru kursa\\Tehničko i naučno pisanje\\ Matematički fakultet}}

\author{Lea Kojičić, \\ mr21079@matf.bg.ac.rs \and
        Isidora Jevremović \\ mi22@matf.bg.ac.rs \and
        Filip Erak \\  mi22@gmail.com \and
        Sara Gojaković \\ mi22244@matf.bg.ac.rs}
\date{\today}
\maketitle

\abstract{U ovom seminarskom radu razmatrano je pitanje Da li veštačka inteligencija može zameniti ljudsko postojanje. Od samog nastanka računara ljude je zanimalo da li će mašine jednog dana moći da nas zamene. 

\tableofcontents

\newpage

\section{Uvod}
\label{poglavlje:uvod}
Mogućnost stvaranja inteligentnih mašina zaokuplja ljudsku maštu još od davnih vremena. Tek sada, sa brzim tempom razvoja računara i već pedesetogodišnjim iskustvom na polju istraživanja tehnika programiranja. San o pametnim mašinama počeo je da postaje stvarnost.\\
Da li ste se nekada zapitali kako je počeo razvoj računara koje mi svakodnevno koristimo? Alan Tjuring je jedan od blistavih umova kome treba da zahvalimo.  Njegov rad je bio uvod u moderan računarski svet i prva vizija pojma veštačke inteligencije (eng. artifical intelligence AI). U Drugom svetskom ratu se njegova zasluga ogleda u dešifrovanju Enigme, mašine koju je nemačka armija tada koristila radi sigurnog slanja šifrovanih poruka. Alan Tjuring je bio fasciniran inteligencijom i razmišljanjem te je i osmislio test 1950. godine poznat pod nazivom "Tjuringov test", koji je uvod u razvoj veštačke inteligencije. Njegovo pitanje je bilo da li mašina može da razmišlja pametno kao i čovek. Danas posle toliko godina uz napredak veštačke inteligencije idalje se postavlja isto pitanje. Ljudi prepoznaju današnje računare kao inteligentne jer imaju potencijal da uče i odlučuju na osnovu informacija koje su im date. Međutim pored svih prednosti koje veštačka inteligencija pruža, postoje stvari u kojima ona ne može da zameni čoveka.\\ 
Postoji razlika između „jake“ veštačke inteligencije(eng. strong AI) — kompjuterske funkcije koje zaista imaju snažne sličnosti sa inteligentnim ljudskim rasuđivanjem i pokazuju neku vrstu  svesti  i „slaba“ veštačka inteligencija (eng. weak A.)— računarske aplikacije koje se bave ograničenim oblastima primene i sadrže neka praktična znanja. 

\newpage








\end{document}