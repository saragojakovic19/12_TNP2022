% !TEX encoding = UTF-8 Unicode

\documentclass[a4paper]{article}

\usepackage{color}
\usepackage{url}
\usepackage[utf8]{inputenc} 


\usepackage[english,serbian]{babel}

\usepackage[unicode]{hyperref}
\hypersetup{colorlinks,citecolor=green,filecolor=green,linkcolor=blue,urlcolor=blue}


%\addbibresource{Literatura.bib} 
\begin{document}

\title{Da li Veštačka inteligencija može da zameni ljudsko postojanje ?\\ 
\small{Seminarski rad u okviru kursa\\Tehničko i naučno pisanje\\ Matematički fakultet}}

\author{Lea Kojičić, \\ mr21079@matf.bg.ac.rs \and
        Isidora Jevremović \\ mi22@matf.bg.ac.rs \and
        Filip Erak \\  mi22@gmail.com \and
        Sara Gojaković \\ mi22244@matf.bg.ac.rs}
\date{\today}
\maketitle

\abstract{U ovom seminarskom radu razmatrano je pitanje Da li veštačka inteligencija može zameniti ljudsko postojanje. Od samog nastanka računara ljude je zanimalo da li će mašine jednog dana moći da nas zamene. 

\tableofcontents

\newpage

\section{Uvod}
\label{poglavlje:uvod}
Mogućnost stvaranja inteligentnih mašina zaokuplja ljudsku maštu još od davnih vremena. Tek sada, sa brzim tempom razvoja računara i već pedesetogodišnjim iskustvom na polju istraživanja tehnika programiranja. San o pametnim mašinama počeo je da postaje stvarnost.\\
Da li ste se nekada zapitali kako je počeo razvoj računara koje mi svakodnevno koristimo? Alan Tjuring je jedan od blistavih umova kome treba da zahvalimo.  Njegov rad je bio uvod u moderan računarski svet i prva vizija pojma veštačke inteligencije (eng. artifical intelligence AI). U Drugom svetskom ratu se njegova zasluga ogleda u dešifrovanju Enigme, mašine koju je nemačka armija tada koristila radi sigurnog slanja šifrovanih poruka. Alan Tjuring je bio fasciniran inteligencijom i razmišljanjem te je i osmislio test 1950. godine poznat pod nazivom "Tjuringov test", koji je uvod u razvoj veštačke inteligencije. Njegovo pitanje je bilo da li mašina može da razmišlja pametno kao i čovek. Danas posle toliko godina uz napredak veštačke inteligencije idalje se postavlja isto pitanje. Ljudi prepoznaju današnje računare kao inteligentne jer imaju potencijal da uče i odlučuju na osnovu informacija koje su im date. Međutim pored svih prednosti koje veštačka inteligencija pruža, postoje stvari u kojima ona ne može da zameni čoveka.\\ 
Postoji razlika između „jake“ veštačke inteligencije(eng. strong AI) — kompjuterske funkcije koje zaista imaju snažne sličnosti sa inteligentnim ljudskim rasuđivanjem i pokazuju neku vrstu  svesti  i „slaba“ veštačka inteligencija (eng. weak A.)— računarske aplikacije koje se bave ograničenim oblastima primene i sadrže neka praktična znanja. 

\newpage


\documentclass{article}

% Language setting
% Replace `english' with e.g. `spanish' to change the document language
\usepackage[english]{babel}

% Set page size and margins
% Replace `letterpaper' with `a4paper' for UK/EU standard size
\usepackage[letterpaper,top=2cm,bottom=2cm,left=3cm,right=3cm,marginparwidth=1.75cm]{geometry}

% Useful packages
\usepackage{amsmath}
\usepackage{graphicx}
\usepackage[colorlinks=true, allcolors=blue]{hyperref}

\title{Neke od primena veštačke inteligencije}
\author{}

\begin{document}
\maketitle


\section{Uvod}

Upoznati smo sa time da danas roboti zamenjuju ljude u mnogim poslovima, obavljaju razne zadatke koji zahtevaju razmišljanje i učenje, rešavaju probleme, pa čak i donose odluke. Veruje se da će veštačka inteligencija postati dominantan oblik inteligencije na zemlji i zameniti većinu ljudskih poslova, samim tim učiniti ljude suvišnim. Da li je taj scenario realan ili ipak postoje neke oblasti u kojima su ljudi nezamenljivi?

\section{Veštačka inteligencija umesto čoveka}

\subsection{Veštačka inteligencija u saobraćaju}

U poslednjoj deceniji došlo je do značajnog unapredjenja tehnologije samovozećih automobila. Ove nove sposobnosti će imati veliki globalni uticaj koji bi mogao značajno da promeni društvo, a da ne pominjemo značajna poboljšanja koja donose opštoj efikasnosti, pogodnosti i bezbednosti naših puteva i transportnih sistema. Rešavanje problema vezanih za tehnologiju samostalnog upravljanja je važno, posebno imajući u vidu široke potencijalne uticaje. Širom sveta se godišnje pređe 10 triliona automobilskih milja, sa složenim i novim uslovima koji stvaraju milione situacija u kojima bi autonomna vozila mogla da pokvare. Ipak, postoje mnogi izazovi koji ostaju na svim nivoima funkcionalnosti sistema.

Takodje u vodnom saobraćaju, komitet za pomorsku sigurnost Medjunarodne pomorske organizacije (the International Maritime Organization – IMO) je jos 2017. godine prihvatio izazov u pripremi pravnog okvira za uvodjenje autonomnih brodova – brodova bez posade. U pripremi je pravlinik u kojem su već definisana tehnička pravila na osnovu kojih bi oni radili. 

Naravno, takve i slične inovacije nisu česta pojava samo u automobilskoj i brodo-mašinskoj industriji. Danas možemo čuti i za avione kojima pilotira veštačka inteligencija i mnoga druga prevozna sredstva kojima će u budućnosti upravljati roboti.

Pomoću AI prevozna sredstva će postati mnogo efikasnija, imaće povećani kapacitet i bolje planiranje ruta. Na primer, često se dešava da vozovi do odredišta stižu puni a vraćaju se prazni. Ovo je jako štetno za ekonomiju i životnu sredinu. 

Takodje, ono što je najvažnije, došlo bi do potpune eliminacije ljudske greške. Veliki broj nesreća na putu je prouzrokovan upravo umorom, nepažnjom ili zbog neodgovornosti ljudi. 

\subsection{Veštačka inteligencija u okviru arheološke nauke}

Datiranje predmeta informatičkim sredstvima, postalo je potreba koja se može rešiti samo kompleksnim ekspertnim sistemom. Veštačka inteligencija u službi arheologije pruža široke mogućnosti koje nisu ostvarive klasičnim arheološkim sistemima dokumentovanja i obrade.
Ekspertni sistemi su programi koji manipulišu znanjem iz neke oblasti da bi na kvalitetan način odgovarali na pitanja koja uobičajeno rešavaju ljudi - eksperti. Primene ekspertnih sistema su česte u medicini, hemiji, vojnoj i naftnoj industriji. Međutim, retki su primeri ekspertnih sistema u društvenim naukama, pa i u arheologiji.


\subsubsection{Ekspertni sistem PANDORA i rezultati rada}


U ovom trenutku, ekspertni sistem PANDORA poseduje oko 600 pravila o rimskim lampama. Kvalitet odgovora koji daje PANDORA, kada su joj dostupni relevantni podaci, je na nivou arheologa eksperta koji se bavi ovom oblašću. Istovremeno, PANDORA je u stanju i da postavlja,
u odnosu na literaturu, samostalne hipoteze o datiranju iskopina. Na svakom koraku izvođenja, PANDORA na zahtev korisnika nudi objašnjenje u vezi istorije rada, odnosno razloga za neki postupak. Multimedijalne pogodnosti, prikazivanje slika, video i zvučnih zapisa, kao i svojevrsna baza podataka znatno brže dovode do relevantnih podataka, nego što je to slučaj sa klasičnim tekstom. Pored toga, nezavisnost baze znanja od mehanizma izvođenja dozvoljava da isti program radeći
nad raznim podacima preuzima ulogu eksperata iz raznih oblasti. Sve ovo čini da je PANDORA
pogodna i za uloge konsultanta istraživačima i kao edukativno sredstvo kojim se studentima il-
ustruje rad eksperata. Odgovori koje daje PANDORA su implicitno sadržani u znanju, odnosno pravilima u bazi znanja. U tom smislu, PANDORA ne može ponuditi ništa
što joj na posredan način nije unapred ugrađeno. Ali, PANDORA sistematski provera mogućnosti
i može doći do odgovora koje bi čovek prevideo zbog velikog obima raspoloživog znanja. Na taj
način PANDORA može davati odgovore koji do sada nisu ponuđeni u stručnoj literaturi.


\subsection{Veštačka inteligencija u psihijatriji}

Mnogi tvrde da će AI transformisati tehničke aspekte psihijatrije. Na primer, tehnologija pametnog telefona omogućava prikupljanje podataka, čija analiza obećava znatno poboljšanu karakterizaciju bolesti i njihovih putanja. Takve metode su već pokazale potencijal u predviđanju recidiva kod bipolarnog poremećeja. Zajedno sa poboljšanim modelima lečenja efikasnost i sve prirodnije taksonomije mentalnih bolesti zasnovane na podacima, čini se verovatno da će kompjuteri u dijagnozi i planiranju lečenja uskoro nadmašiti ljude.

AI još uvek nije u stanju da razgovara sa dovoljno fleksibilnosti da bi održao psihijatrijski intervju. Međutim, obrada prirodnog jezika brzo napreduje i konverzacijski agenti su već našli primenu u proceni navika konzumiranja alkohola. Zaista, postoje jaki dokazi koji sugerišu da ljudi mogu da izgrade terapeutske veze sa agensima veštačke inteligencije. Dokazi sugerišu da ljudi mogu biti iskreniji prema računarima nego prema ljudima. Čini se vrlo verovatno da će ljudi lako doživeti AI kliničara kao istinski brižnog i razumevajućeg. Uz dovoljno podataka, buduća AI će izgraditi dovoljno dubok model odgovora osobe da njeno razumevanje njih prevazilazi razumevanje njihovog psihijatra.

Možda će ljudski pacijenti želeti ljudske lekare, sa svim njihovim čudnostima i komparativnom nesposobnošću, jednostavno zato što su ljudi. Cinično, ako su silicijumski psihijatri jeftiniji i merljivo efikasni kao ljudski psihijatri, mogli bi naći masovno zaposlenje samo na osnovu ekonomskih zahteva. Još pozitivnije, AI obećava značajne prednosti za pacijente. Umesto da složene pojedince svrstava u dijagnostičke kategorije i da dodeljuje tretmane zasnovane na generičkim smernicama, AI nudi zaista individualizovanu negu. Štaviše, za razliku od veoma fragmentiranih puteva nege kojima su pacijenti trenutno izloženi, lični AI kliničar bi bio dostupan manje-više bilo gde (od primarne nege do odeljenja za pacijente), u bilo koje vreme. Vremenom bi se tada moglo zaraditi i održati poverenje pacijenata. Stoga, ljudi mogu smatrati da je briga vođena veštačkom inteligencijom humanija od statusa današnje psihijatrije; u svojoj želji da budu shvaćeni i tretirani na osnovu najboljeg naučnog razumevanja, oni će voljno izabrati AI umesto njegove alternative od krvi i mesa.

\bibliographystyle{alpha}
\bibliography{sample}
\begin{itemize}
\item 'Will artificial intelligence eventually replace psychiatrists?'
Published online by Cambridge University Press: 06 December 2019
Christian Brown , Giles William Story , Janaina Mourão-Miranda and Justin Taylor Baker,
\item Srbobran Branković FEFA, Beograd; Filozofski fakultet, Niš 'VEŠTAČKA INTELIGENCIJA I DRUŠTVO' Srpska politička misao, broj 2/2017.god. Institut za političke studije, Beograd, 24. vol. 56. str. 13-32.
\item 'Artificial Intelligence and its Role in Near Future' Jahanzaib Shabbir, and Tarique Anwer
\end{itemize}

\newpage





\end{document}
