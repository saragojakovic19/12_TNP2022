\documentclass{beamer}
\usepackage{beamerthemeshadow}
\usepackage{graphicx}
\usepackage{color}
\usepackage[utf8]{inputenc}
\usepackage{hyperref}
\usepackage{caption}
\usepackage[flushleft]{threeparttable}
\usepackage{subfigure}
\definecolor{bluebell}{rgb}{0.64, 0.64, 0.82}
\setbeamercolor{structure}{fg=bluebell}
\captionsetup[figure]{labelformat=empty}
\captionsetup[table]{labelformat=empty}


\def\d{{\fontencoding{T1}\selectfont\dj}}
\def\D{{\fontencoding{T1}\selectfont\DJ}}


\title{Tehničko i naučno pisanje}
\subtitle{Da li veštačka inteligencija može da zameni ljudsko postojanje?}
\author{Isidora Jevremović \and Filip Erak\ \and Lea Kojičić \and Sara Gojaković}
\institute{Matematički fakultet\\Univerzitet u Beogradu}
\date{
	\footnotesize{Beograd, 2022.}	
}

\begin{document}
\begin{frame}
	\thispagestyle{empty}
	\titlepage
\end{frame}

\addtocounter{framenumber}{-1}

\begin{frame}[fragile]\frametitle{Literatura}
	\begin{itemize}
		\item Seminarski rad koji smo radili na ovu temu možete naći na sledećem linku:
		(\url{https://github.com/saragojakovic19/12_TNP2022/blob/main/12_JevremovicKojicicErakGojakovic.pdf})
	\end{itemize}
\end{frame}

\begin{frame}
	\frametitle{Pregled} % Table of contents slide, comment this block out to remove it
	\tableofcontents[] 
\end{frame}
\section{Da li veštačka inteligencija može da zameni ljudsko postojanje?}

\subsection{Uvod}

\begin{frame}[fragile]\frametitle{Uvod}
	\begin{itemize}	
		\item Rad Alana Tjuringa je bio uvod u moderan računarski svet i vizija pojma veštačke inteligencije.
		\item Pored svih prednosti koje veštačka inteligencija pruža postoje stvari u kojima ona ne može zameniti čoveka.
       \item Razlikujemo dve vrste veštačke inteligencije, jaku i slabu.
	\end{itemize}
 

 
\end{frame}


\begin{frame}[fragile]\frametitle{Klasifikacija na osnovu IQ testa}
\label{tab:tabelaIQ}
\begin{tabular}{|c|c|c|} \hline
IQ Domen & Klasifikacija & Procenat ljudske populacije\\ \hline
(55, 70]& Granična zaostalost & 2\%\\ \hline
(70, 85] & Zatupljenost & 14\%\\ \hline
(85, 115] & Prosečna inteligencija & 68\%\\ \hline
(115, 130] & Visok nivo inteligencije & 14\%\\ \hline
(130, 145] & Veoma visok nivo & 2\% \\ \hline
\end{tabular}

\end{frame}



\end{document}
